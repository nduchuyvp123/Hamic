\documentclass[10pt]{article}
\usepackage{vntex}
\usepackage{tikz}
\usepackage[left=3.00cm, right=2.00cm, top=2.00cm, bottom=2.00cm]{geometry}
\usepackage[unicode]{hyperref}
\usepackage{amsmath}
\usepackage{amssymb}
\usepackage{graphicx}
\usepackage{a4wide,amssymb,epsfig,latexsym,array,hhline,fancyhdr}
\usepackage[normalem]{ulem}
\usepackage[makeroom]{cancel}
\usepackage{amsthm}
\usepackage{multicol,longtable,amscd}
\usepackage{diagbox}
\usepackage{booktabs}
\usepackage{alltt}
\usepackage[framemethod=tikz]{mdframed}
\usepackage{caption,subcaption}
\usepackage[left=3.00cm, right=2.00cm, top=2.00cm, bottom=2.00cm]{geometry}
\usepackage{listings}
\usepackage{color}
\usepackage{lipsum}
\usepackage{setspace}
\usepackage{titling}
\usepackage{indentfirst}

\usetikzlibrary{decorations}
\usetikzlibrary{decorations.pathreplacing}
\usetikzlibrary{decorations.pathreplacing,calligraphy}

\setstretch{1}

\newtheorem{theorem}{Định lý}
\newtheorem{corollary}{Hệ quả}
\newtheorem{lemma}{Bổ đề}
\newtheorem*{remark}{Nhận xét}
\newtheorem{definition}{Định nghĩa}


\title{\textbf{Kuratowski's Theorem}}
\posttitle{
\par\end{center}
\begin{center}\LARGE(Toán rời rạc)\end{center}
\vskip0.5em}



\author{
    Nguyễn Đức Huy \\
    Departement \\
    Đại học Khoa học Tự Nhiên \\
    mail@edu
    \and
    Trần Thị Như Quỳnh \\
    Departement \\
    Đại học Khoa học Tự Nhiên \\
    mail@edu
    \and
    Trần Khánh Duy \\
    Departement \\
    Đại học Khoa học Tự Nhiên \\
    mail@edu
}

\begin{document}
\maketitle

\section{Introduction}
Theorems can easily be defined

\begin{theorem}
    Let $f$ be a function whose derivative exists in every point, then $f$ is
    a continuous function.
\end{theorem}

\begin{theorem}[Pythagorean theorem]
    \label{pythagorean}
    This is a theorema about right triangles and can be summarised in the next
    equation
    \[ x^2 + y^2 = z^2 \]
\end{theorem}

And a consequence of theorem \ref{pythagorean} is the statement in the next
corollary.

\begin{corollary}
    There's no right rectangle whose sides measure 3cm, 4cm, and 6cm.
\end{corollary}

You can reference theorems such as \ref{pythagorean} when a label is assigned.

\begin{lemma}
    Given two line segments whose lengths are $a$ and $b$ respectively there is a
    real number $r$ such that $b=ra$.
\end{lemma}
Unnumbered theorem-like environments are also posible.

\begin{remark}
    This statement is true, I guess.
\end{remark}
\section{Second Section}
And the next is a somewhat informal definition

\begin{definition}[Fibration]
    A fibration is a mapping between two topological spaces that has the homotopy lifting property for every space $X$.
\end{definition}


\begin{lemma}
    Given two line segments whose lengths are $a$ and $b$ respectively there
    is a real number $r$ such that $b=ra$.
\end{lemma}

\begin{proof}
    To prove it by contradiction try and assume that the statement is false,
    proceed from there and at some point you will arrive to a contradiction.
\end{proof}
\end{document}